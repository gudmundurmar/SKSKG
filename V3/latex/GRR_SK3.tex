\documentclass{article}
% Útgáfa 1.5

% pakkar fyrir töflur; array er notað í "math-mode"; arydshln fyrir brotalínur í töflum
\usepackage{array,tabularx}%,arydshln}  
% íslenskt letur, orðaskiptingar ...
\usepackage[icelandic]{babel}
\usepackage[T1]{fontenc}
% númera jöfnur, töflur, myndir, ...
\usepackage{enumerate}
% fyrir krækjur
\usepackage[colorlinks,linkcolor=blue,citecolor=blue,urlcolor=blue]{hyperref}
% ýmis tákn, leturgerðir, ... ATH amsmath fyrir \text{} skipunina
\usepackage{amsmath,amssymb,euscript}   
% til að setja inn *.eps myndir ef við notum dvi og ps skjöl...
\usepackage{epsfig}    % \epsfig{...}
% ... en ef við notum pdf beint, þá er þetta pakkinn sem við þurfum
\usepackage{graphicx}  % \includegraphics[...]{...}
% litamöguleikar fyrir texta
\usepackage{color}
% til þess að myndir og töflur standi þar sem þær eiga að standa!
\usepackage{here}
%
\setlength{\textwidth}{6in}
\setlength{\textheight}{9in}
\setlength{\headheight}{0in}
\setlength{\headsep}{0in}
\setlength{\topskip}{0in}
\setlength{\topmargin}{0cm}
\setlength{\oddsidemargin}{0in}
% dregur fyrstu línu í hverri málsgrein inn - nota 0cm fyrir engan inndrátt fyrir allar málsgreinar en \noindent í upphafi málsgreinar til að hafa engan inndrátt í þeirri málsgrein einni:
\setlength{\parindent}{0cm}
% viljum hafa eitt línubil milli efnisgreina
\setlength{\parskip}{1.5ex plus 0.75ex minus 0.5ex}
% 1.5 í línubil
\renewcommand{\baselinestretch}{1.025}

% Environments
\newenvironment{alist}[1][$\quad\,$1.]{
\vspace*{-8pt} \begin{enumerate}[label=\alph*),itemsep=4pt,parsep=3pt]}
{\end{enumerate}\vspace*{-3pt}}

\newenvironment{ttafla}[1][$\quad\,$1.]{
\begin{tabular}{rcl} \renewcommand\arraystretch{2} }
{\end{tabular}}



%Bætt við af mér (Hannesi)
% Þjappa saman
\widowpenalty=600
\clubpenalty=600
\usepackage[compact]{titlesec}
\titlespacing{\section}{0pt}{2ex plus 0.75ex minus 0.75ex}{1ex plus 0.3ex minus 0.3ex}
\titlespacing{\subsection}{0pt}{1ex plus 0.25ex minus 0.25ex}{0ex plus 0.1ex minus 0.1ex}
\titlespacing{\subsubsection}{0pt}{0.5ex plus 0.1ex minus 0.1ex}{0ex plus 0.1ex minus 0.1ex}

% Bolda caption
\usepackage[hang,small,bf]{caption}

% Efnaformúlur og myndir
%\usepackage[version=3]{mhchem}
\usepackage{chemfig}

% Komma í stað punkts
\usepackage{icomma}

% Annað
\usepackage{subfig}
\usepackage{array}
\usepackage{bigstrut}
\usepackage{multirow}
\usepackage{multicol}
\usepackage{enumerate}
%\usepackage{enumitem}
\usepackage{wrapfig}
\newcommand{\HRule}{\rule{\linewidth}{0.5mm}}
\usepackage{ulem}
\usepackage[thinspace,mediumqspace,amssymb]{SIunits}



\usepackage[utf8]{inputenc}
\usepackage{minted}


\begin{document}

\begin {titlepage}
\begin{center}
\includegraphics[width=0.15\textwidth]{./Haskoli_Islands_rett.jpg}~\\[1cm]

\textsc{\LARGE Greining Reiknirita}\\[1.5cm]

\textsc{\Large TÖL403G}\\[0.5cm]

% Title
\HRule \\[0.4cm]
{ \huge \bfseries Skilaverkefni 3\\[0.4cm] }

\HRule \\[1.5cm]

% Author and supervisor
\begin{minipage}{0.5\textwidth}
\begin{flushleft} \large
\emph{Verkefnishöfundar:}
\\ Guðmundur\textsc{ Már Gunnarsson }
\\ Skarphéðinn \textsc{Þórðarsson}
\\ Sigurður \textsc{Skúli Sigurgeirsson}

\end{flushleft}
\end{minipage}
\begin{minipage}{0.4\textwidth}
\begin{flushright} \large
\emph{Kennari:} \\
Páll \textsc{Melsted}
\end{flushright}
\end{minipage}

\vfill

% Bottom of the page
{\large \today}

\end{center}

\end{titlepage}

\section {Keyrsla og virkni}
Öll forritun fór fram í Python og notast við Python version 2.7 fyrir Windows við keyrslu þess. 
Forritið er aðgengilegt á \url{https://github.com/gudmundurmar/SKSKG/V3}
Þar er einnig að finna Stopwatch.java sem notast var við til þess að taka timann á forritinu en einnig uppgefin inntök inntök og úttök s1-s3 sem notast var við til þess að keyra forritið líkt og gefið var upp í verkefnislýsingunni. 

 	Keyrsla: \texttt {\$ Python verk3.py < 10.in | diff -w 10.out -}


\section {Keyrslutími}

Tími: Aðferðin doubleBFS skoðar í mesta lagi alla hnúta í netinu einu sinni og tekur því O(V). Það að finna rétta hnútinn sem tengir saman samhengisþættina tekur O(V*(V-E)). Það þarf að keyra í mesta lagi V-E sinnum í gegnum forlykkju því það er lengdin á notSpan og lengsta mögulega lengdin á context (samhengisþættinum) er V/2 og af því að við gerum “if notSpan[cur][1] in context” sem tekur O(V).
Lykkjuna til að finna N-1 spantrén keyrum við O(V) sinnum svo að tímaflækjan fyrir  næstminnsta spantré fyrir hvern legg er O(V*(V+V*(E-V)))


\begin{center}
    \begin{tabular}{| l | l | l | l |}
    \hline
    Inntak & Keyrslutími\\ \hline
    10.in & tala\\ \hline
    100.in & tala\\ \hline
    1k.in & tala\\ \hline
    10k & 117.254707098 sek\\ \hline
    100k & Keyrslutími\\ \hline

    \end{tabular}
\end{center}
117.254707098 seconds


\subsection{Meginklasinn - Verk3}
\inputminted{python}{verk3.py}


\subsection{Priority dict}
\inputminted{python}{priority_dict.py}


\pagebreak

\section {Forritið í heild}

\inputminted[mathescape,
               linenos,
               numbersep=10pt,
               gobble=0,
               frame=lines,
               framesep=2mm]{python}{../verk3.py}


\end{document}