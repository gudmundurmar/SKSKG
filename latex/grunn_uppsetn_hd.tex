% Útgáfa 1.5

% pakkar fyrir töflur; array er notað í "math-mode"; arydshln fyrir brotalínur í töflum
\usepackage{array,tabularx}%,arydshln}  
% íslenskt letur, orðaskiptingar ...
\usepackage[icelandic]{babel}
\usepackage[T1]{fontenc}
% númera jöfnur, töflur, myndir, ...
\usepackage{enumerate}
% fyrir krækjur
\usepackage[colorlinks,linkcolor=blue,citecolor=blue,urlcolor=blue]{hyperref}
% ýmis tákn, leturgerðir, ... ATH amsmath fyrir \text{} skipunina
\usepackage{amsmath,amssymb,euscript}   
% til að setja inn *.eps myndir ef við notum dvi og ps skjöl...
\usepackage{epsfig}    % \epsfig{...}
% ... en ef við notum pdf beint, þá er þetta pakkinn sem við þurfum
\usepackage{graphicx}  % \includegraphics[...]{...}
% litamöguleikar fyrir texta
\usepackage{color}
% til þess að myndir og töflur standi þar sem þær eiga að standa!
\usepackage{here}
%
\setlength{\textwidth}{6in}
\setlength{\textheight}{9in}
\setlength{\headheight}{0in}
\setlength{\headsep}{0in}
\setlength{\topskip}{0in}
\setlength{\topmargin}{0cm}
\setlength{\oddsidemargin}{0in}
% dregur fyrstu línu í hverri málsgrein inn - nota 0cm fyrir engan inndrátt fyrir allar málsgreinar en \noindent í upphafi málsgreinar til að hafa engan inndrátt í þeirri málsgrein einni:
\setlength{\parindent}{0cm}
% viljum hafa eitt línubil milli efnisgreina
\setlength{\parskip}{1.5ex plus 0.75ex minus 0.5ex}
% 1.5 í línubil
\renewcommand{\baselinestretch}{1.025}

% Environments
\newenvironment{alist}[1][$\quad\,$1.]{
\vspace*{-8pt} \begin{enumerate}[label=\alph*),itemsep=4pt,parsep=3pt]}
{\end{enumerate}\vspace*{-3pt}}

\newenvironment{ttafla}[1][$\quad\,$1.]{
\begin{tabular}{rcl} \renewcommand\arraystretch{2} }
{\end{tabular}}



%Bætt við af mér (Hannesi)
% Þjappa saman
\widowpenalty=600
\clubpenalty=600
\usepackage[compact]{titlesec}
\titlespacing{\section}{0pt}{2ex plus 0.75ex minus 0.75ex}{1ex plus 0.3ex minus 0.3ex}
\titlespacing{\subsection}{0pt}{1ex plus 0.25ex minus 0.25ex}{0ex plus 0.1ex minus 0.1ex}
\titlespacing{\subsubsection}{0pt}{0.5ex plus 0.1ex minus 0.1ex}{0ex plus 0.1ex minus 0.1ex}

% Bolda caption
\usepackage[hang,small,bf]{caption}

% Efnaformúlur og myndir
%\usepackage[version=3]{mhchem}
\usepackage{chemfig}

% Komma í stað punkts
\usepackage{icomma}

% Annað
\usepackage{subfig}
\usepackage{array}
\usepackage{bigstrut}
\usepackage{multirow}
\usepackage{multicol}
\usepackage{enumerate}
%\usepackage{enumitem}
\usepackage{wrapfig}
\newcommand{\HRule}{\rule{\linewidth}{0.5mm}}
\usepackage{ulem}
\usepackage[thinspace,mediumqspace,amssymb]{SIunits}


